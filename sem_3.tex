%% Written by Igor Pinaev, SPbSTU, 2016

\begin{center}
  {\large \bf III семестр}
  \rule{\textwidth}{1pt}
\end{center}

\begin{utv}
  $C_{0}^{\infty}(\Omega)$ плотно в $L^p(\Omega)$
\end{utv}

\begin{definition}[Усреднение]
$\letus \, u \in L^1(\Omega), \; \Omega \subset \mathbb{R}^n$ -- ограниченная область\\
$$u_\rho(x) = \int\limits_{\Omega} \omega_\rho(x-y)u(y)dy, \, \forall x \in \mathbb{R}^n$$
$u_\rho(x)$ -- усреднение функции
\end{definition}

\begin{lemma}[Дюбуа-Раймонда]
 Пусть \begin{minipage}[t]{0.8\linewidth}\begin{enumerate}[itemsep=1mm]
    \item $u \in L^1_{loc}(\Omega)$
    \item $\int\limits_\Omega uq dx = 0, \; \forall q \in C_0^\infty(\Omega)$
    \end{enumerate}\end{minipage}\\
  %$u \in L^1_{loc}(\Omega)$ и $\int\limits_\Omega uq dx = 0, \; \forall q \in C_0^\infty(\Omega)$ \\
  Тогда $u = 0$ почти всюду в $\Omega$
\end{lemma}

\begin{definition}
  $u \in L^1_{loc}(\Omega)$, если $u \in L^1(K)$ $\forall K \subset \Omega$
\end{definition}

\begin{definition}[Компактное вложение]
  Пусть \begin{minipage}[t]{0.8\linewidth}\begin{enumerate}[itemsep=1mm]
    \item $\overline{\Omega'}$ -- компакт
    \item $\overline{\Omega'} \subset \Omega$
  \end{enumerate}\end{minipage}\\

  \noindent Тогда $\Omega' \subset\subset \Omega$ -- компактное вложение $\Omega'$ в $\Omega$
\end{definition}

\begin{theorem}[Рисс]
  Пусть $F$ -- предкомпакт в $L^p(\Omega)$
  $\Longleftrightarrow$ \begin{minipage}[t]{0.8\linewidth}\begin{enumerate}[itemsep=1mm]
      \item $\exists M \ge 0: ||u||_{L^p(\Omega)} \le M, \; \forall u \in F$
      \item $\sup\limits_{|z| < \rho}\sup\limits_{u \in F} ||u(x+z)-u(x)||_{L^p(\Omega)} = f(\rho) \underset{\rho \rightarrow 0}{\longrightarrow} 0$
  \end{enumerate}\end{minipage}
      
\end{theorem}
